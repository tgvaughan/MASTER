\documentclass[11pt]{article}

\usepackage{amsmath, amsfonts}

\begin{document}


\title{Stochastic simulation of non-linear epidemiology with sampling}
\author{Alexei Drummond}
\date{}
\maketitle


\section{Nonlinear epidemiology with sampling}

Consider the following coupled pair of processes:

\begin{eqnarray}
I + S & \overset{\beta} \longrightarrow & 2I \nonumber \\
I & \overset{\gamma}\longrightarrow & R,
\end{eqnarray}

This is the reaction scheme for a continuous-time discrete-state analogue of the classic deterministic Kermack-McKendrick SIR (susceptible-infected-recovered) model of a closed epidemic.

Lets add a third process to differentiate between observed and unobserved recoveries:

\begin{eqnarray}
I + S & \overset{\beta} \longrightarrow & 2I \nonumber \\
I & \overset{(1-s)\gamma}\longrightarrow & R_h \\
I & \overset{s\gamma}\longrightarrow & R_s,
\end{eqnarray}

We now distinguish between {\it hidden} or unobserved recoveries $R_h$ and {\it sampled} or observed recoveries $R_s$. This censored infection process, where only some recoveries are observed, is the basis for connecting non-linear epidemiological models to phylogenetic data. The new parameter $s$ is the {\it sampling proportion}, i.e. the probability of a recovery being observed, and thus the expected proportion of recoveries observed.



\end{document}
